$$
%
Discrete Diffusion
%
$$

This one was for homework, but I thought it was cool I'm postin' it!

Consider a spatial interval $[a, b]$. Let $M$ be a given positive integer and define the spatial step size $h = (b - a)/M$.<br>
The spatial grid points are then defined as $x_i = a + i h$, for $0 \le i \le M$. <br>
Similarly, define the time step size $\tau > 0$ and the discrete time levels $t_n = n\tau$, for $0 \le n \le N$. <br>
Let $u_i^n$ denote the numerical approximation to the solution $u(x_i, t_n)$ at the grid point $(x_i, t_n)$.  <br>

The discrete diffusion equation on this space–time lattice is given by

$$u_i^{\,n+1} = u_i^{\,n} + \frac{\tau}{h^2} \left( u_{i+1}^{\,n} - 2u_i^{\,n} + u_{i-1}^{\,n} \right) + f(x_i, t_n), 
\qquad 1 \le i \le M - 1,\quad 0 \le n \le N - 1,$$
subject to the boundary conditions
$$u_0^{\,n} = u_M^{\,n} = 0, \qquad 0 \le n \le N,$$
and the initial condition
$$u_i^{\,0} = g(x_i), \qquad 0 \le i \le M.$$
Here, $f(x_i, t_n)$ is a given source term and $g(x)$ is the given initial condition.

The goal is to perform this computation using matrix multiplications to get from one timestep to the next.

## My Solution

We want to express the discrete diffusion equation as a matrix transformation.

First, let's factor in terms of components of $u^n$
$$u_i^{n+1} = \frac{\tau}{h^2}u_{i+1}^n + (1-\frac{2\tau}{h^2})u_i^n + \frac{\tau}{h^2}u_{i-1}^n + f(x_i,t_n)$$
For clarity, substitute $c = \frac{\tau}{h^2}, d = (1-\frac{2\tau}{h^2})$
$$u_i^{n+1} = cu_{i+1}^n + du_i^n + cu_{i-1}^n + f(x_i,t_n)$$
Steps $u_i^{n+1}$ can be expressed as dot products
$$
u_i^{n+1} =
\begin{bmatrix}
0 & \dots & 0 & c & d & c & 0 & \dots & 0
\end{bmatrix}
\begin{bmatrix}
u_0^n \\ u_1^n \\ \vdots \\ u_{M}^n
\end{bmatrix} + f(x_i,t_n)
$$
Steps at "beginning" and "end" of the lattice $u_0^{n+1}, u_{M}^{n+1}$ are special cases
$$
u_0^{n+1} =
\begin{bmatrix}
d & c & 0 & \dots & 0
\end{bmatrix}
\begin{bmatrix}
u_0^n \\ u_1^n \\ \vdots \\ u_{M}^n
\end{bmatrix} + f(x_0,t_n),\hspace{0.3cm}
u_{M}^{n+1} =
\begin{bmatrix}
0 & \dots & c & d
\end{bmatrix}
\begin{bmatrix}
u_0^n \\ u_1^n \\ \vdots \\ u_{M}^n
\end{bmatrix} + f(x_{M},t_n)
$$
Dot products of the vector $\mathbf{u}^n$ with many different vectors is a matrix multiplication resulting in a vector $\mathbf{u}^{n+1}$.

The source terms can be factored out as a vector addition.
$$
\mathbf{u}^{n+1}=
\begin{bmatrix}
d & c & 0 & 0 & \dots & 0 \\
c & d & c & 0 & \dots & 0 \\
\vdots & \vdots & \vdots & \vdots & \vdots & \vdots \\
0 & \dots & 0 & c & d & c \\
0 & \dots & 0 & 0 & c & d
\end{bmatrix}
\mathbf{u}^n +
\begin{bmatrix}
f(x_0,t_n) \\
\vdots \\
f(x_{M},t_n)
\end{bmatrix}
$$
This matrix is a tridiagonal matrix, which we can construct in numpy as the sum of offset diagonal matrices.